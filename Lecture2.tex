\section{Orthonormal Vectors}
A set of vectors are called orthonormal if they are all of unit length and
mutually perpendicular, i.e.:
$$
	\bm{a}\cdot\bm{b} = 0 \; \forall \; \bm{a}, \bm{b} \in V
$$

\section{Inner Products}
Consider a vector space, $V$. The inner product of two vectors $\bm{a}$,
$\bm{b} \in V$, is denoted by $(\bm{a}, \bm{b})$.

It is a scalar function which satisfies some properties:
\begin{itemize}
	\item $(\bm{a}, \bm{b}) = {(\bm{b}, \bm{a})}^*$for all $\bm{a}, \bm{b} \in V
	$ (symmetry/commutitive)
	\item $(\bm{a}, \lambda\bm{b} + \mu\bm{c}) = 
	\lambda(\bm{a}, \bm{b}) + \mu(\bm{a}, \bm{c})$ for all $\bm{a}, \bm{b},
	\bm{c} \in V$ and $\mu, \lambda \in \mathbb{R}/\mathbb{C}$. (distributive)
\end{itemize}
Two vectors in a general vector space are said to be orthogonal if $(\bm{a},
\bm{b}) = 0$.

\subsection{Dot product}
The dot product is the `standard' inner product that we use on $\mathbb{R}^n$.
It has the value:
$$
	\bm{a}\cdot\bm{b} = |\bm{a}||\bm{b}|\cos\theta
$$
We say that $|\bm{b}|\cos\theta$ is the projection of $\bm{b}$ onto $\bm{a}$.
The value of the dot product can also be found using:
$$
	\bm{a}\cdot\bm{b} = a_1 b_1 + a_2 b_2 + \hdots + a_n b_n
$$

\subsubsection{Complex inner product}
Here we use a generalisation of the dot product:
$$
	<\bm{a}, \bm{b}> = \bm{a}^\dagger\cdot\bm{b} = a_x^*b_x + a_y^*b_x + \hdots
$$
Where the $\dagger$ represents the complex-conjugate transpose of the original
vector.

\subsection{Vector product}
The vector product, rather than returning a scalar, returns a vector. This
vector is perpendicular to both `input' vectors.
$$
	|\bm{a} \times \bm{b}| = |\bm{a}||\bm{b}|\sin\theta
$$
If both vectors are in the same direction (i.e. $\theta=0$) then the vector
product returns 0. To calculate the vector returned:
$$
	\begin{pmatrix}
	a_x\\
	a_y\\
	a_z\\
	\end{pmatrix}
	\times
	\begin{pmatrix}
	b_x\\
	b_y\\
	b_z\\
	\end{pmatrix}
	=
	\begin{pmatrix}
	a_y b_z - a_z b_y\\
	a_z b_x - a_x b_z\\
	a_x b_y - a_y b_x\\
	\end{pmatrix}
$$

\subsubsection{Triple product}
$$
	[\bm{a},\bm{b},\bm{c}] = \bm{a} \cdot (\bm{b} \times \bm{c})
$$

\section{Equation of a line}
Imagine you have a point in space, which you know is on the line ($\bm{a}$).
Then, imagine you have a vector that points in the direction of your line
($\bm{b}$). You can then find any point on the line ($\bm{r}$):
$$
	\bm{r} = \bm{a} + \lambda\bm{b}
$$
However, if you know two points ($\bm{c}$, $\bm{d}$), you can work out the
direction vector:
$$
	\bm{r} = (\bm{c} - \bm{d})\lambda + \bm{c}
$$
It is also possible to describe a line using a vector product. This is because
you know that everything in the line is in the same direction as the direction
vector $\bm{b}$:
$$
	(\bm{r} - \bm{a})\times\bm{b} = 0
$$

\section{Equation of a plane}
A plane goes through a point $\bm{a}$ with a normal vector $\hat{n}$. Any point
in the plane can then be found by using a scalar product (a vector `in' the
plane dotted with the normal is always 0):
$$
	(\bm{r} - \bm{a})\cdot\hat{\bm{n}} = 0
$$
We can re-write this:
$$
	xn_x + yn_y + zn_z = \bm{a}\cdot\hat{\bm{n}} = d
$$
Another way of getting a plane is to specify three points lying in the plane,
$\bm{a}$, $\bm{b}$, $\bm{c}$, and specify vectors that lie in the plane using
these:
$$
	\hat{\bm{n}} = (\bm{b} - \bm{a}) \times (\bm{b} - \bm{c})
$$

\section{Equation of a sphere}
If we know the centre of the sphere, $\bm{c}$, and the radius, $a$, we know that
the distance to any point on the surface ($\bm{r}$) from the centre is $a$:
$$
	|\bm{r} - \bm{c}| = a^2
$$
This expands out to the familar equation:
$$
	{(x-c_x)}^2 + {(y-c_y)}^2 + {(z-c_z)}^2 = a^2
$$

